% \iffalse meta-comment
%
% Copyright (C) 2015 by Ben Lavery
% -----------------------------------
%
% This file may be distributed and/or modified under the
% conditions of the LaTeX Project Public License, either version 1.2 
% of this license or (at your option) any later version.
% The latest version of this license is in:
%
% http://www.latex-project.org/lppl.txt
%
% and version 1.2 or later is part of all distributions of LaTeX
% version 1999/12/01 or later.
%
% \fi
%
% \iffalse
%<*driver>
\ProvidesFile{bil-CV.dtx}
%</driver> %<class>\NeedsTeXFormat{LaTeX2e}[1999/12/01] 
%<class>\ProvidesClass{bil-CV}
%<*class>
[2015/04/01 v1.0 CV template] %</class>
%
%<*driver>
\documentclass{ltxdoc}
\EnableCrossrefs
\CodelineIndex
\RecordChanges
\usepackage{datetime}
\begin{document}
\DocInput{bil-CV.dtx} \end{document}
%</driver>
% \fi
%
% \CheckSum{0}
%
% \CharacterTable
%  {Upper-case    \A\B\C\D\E\F\G\H\I\J\K\L\M\N\O\P\Q\R\S\T\U\V\W\X\Y\Z
%   Lower-case    \a\b\c\d\e\f\g\h\i\j\k\l\m\n\o\p\q\r\s\t\u\v\w\x\y\z
%   Digits        \0\1\2\3\4\5\6\7\8\9
%   Exclamation   \!     Double quote  \"     Hash (number) \#
%   Dollar        \$     Percent       \%     Ampersand     \&
%   Acute accent  \'     Left paren    \(     Right paren   \)
%   Asterisk      \*     Plus          \+     Comma         \,
%   Minus         \-     Point         \.     Solidus       \/
%   Colon         \:     Semicolon     \;     Less than     \<
%   Equals        \=     Greater than  \>     Question mark \?
%   Commercial at \@     Left bracket  \[     Backslash     \\
%   Right bracket \]     Circumflex    \^     Underscore    \_
%   Grave accent  \`     Left brace    \{     Vertical bar  \|
%   Right brace   \}     Tilde         \~}
%
%
% \changes{v1.0}{2015/04/01}{Initial version}
%
% \GetFileInfo{bil-CV.dtx}
\DoNotIndex{\',\.,\@M,\@@input,\@Alph,\@alph,\@addtoreset,\@arabic}
\DoNotIndex{\@badmath,\@centercr,\@cite}
\DoNotIndex{\@dotsep,\@empty,\@float,\@gobble,\@gobbletwo,\@ignoretrue}
\DoNotIndex{\@input,\@ixpt,\@m,\@minus,\@mkboth}
\DoNotIndex{\@ne,\@nil,\@nomath,\@plus,\roman,\@set@topoint}
\DoNotIndex{\@tempboxa,\@tempcnta,\@tempdima,\@tempdimb}
\DoNotIndex{\@tempswafalse,\@tempswatrue,\@viipt,\@viiipt,\@vipt}
\DoNotIndex{\@vpt,\@warning,\@xiipt,\@xipt,\@xivpt,\@xpt,\@xviipt}
\DoNotIndex{\@xxpt,\@xxvpt,\\,\ ,\addpenalty,\addtolength,\addvspace}
\DoNotIndex{\advance,\ast,\begin,\begingroup,\bfseries,\bgroup,\box}
\DoNotIndex{\bullet}
\DoNotIndex{\cdot,\cite,\CodelineIndex,\cr,\day,\DeclareOption}
\DoNotIndex{\def,\DisableCrossrefs,\divide,\DocInput,\documentclass}
\DoNotIndex{\DoNotIndex,\egroup,\ifdim,\else,\fi,\em,\endtrivlist}
\DoNotIndex{\EnableCrossrefs,\end,\end@dblfloat,\end@float,\endgroup}
\DoNotIndex{\endlist,\everycr,\everypar,\ExecuteOptions,\expandafter}
\DoNotIndex{\fbox}
\DoNotIndex{\filedate,\filename,\fileversion,\fontsize,\framebox,\gdef}
\DoNotIndex{\global,\halign,\hangindent,\hbox,\hfil,\hfill,\hrule}
\DoNotIndex{\hsize,\hskip,\hspace,\hss,\if@tempswa,\ifcase,\or,\fi,\fi}
\DoNotIndex{\ifhmode,\ifvmode,\ifnum,\iftrue,\ifx,\fi,\fi,\fi,\fi,\fi}
\DoNotIndex{\input}
\DoNotIndex{\jobname,\kern,\leavevmode,\let,\leftmark}
\DoNotIndex{\list,\llap,\long,\m@ne,\m@th,\mark,\markboth,\markright}
\DoNotIndex{\month,\newcommand,\newcounter,\newenvironment}
\DoNotIndex{\NeedsTeXFormat,\newdimen}
\DoNotIndex{\newlength,\newpage,\nobreak,\noindent,\null,\number}
\DoNotIndex{\numberline,\OldMakeindex,\OnlyDescription,\p@}
\DoNotIndex{\pagestyle,\par,\paragraph,\paragraphmark,\parfillskip}
\DoNotIndex{\penalty,\PrintChanges,\PrintIndex,\ProcessOptions}
\DoNotIndex{\protect,\ProvidesClass,\raggedbottom,\raggedright}
\DoNotIndex{\refstepcounter,\relax,\renewcommand}
\DoNotIndex{\rightmargin,\rightmark,\rightskip,\rlap,\rmfamily}
\DoNotIndex{\secdef,\selectfont,\setbox,\setcounter,\setlength}
\DoNotIndex{\settowidth,\sfcode,\skip,\sloppy,\slshape,\space}
\DoNotIndex{\symbol,\the,\trivlist,\typeout,\tw@,\undefined,\uppercase}
\DoNotIndex{\usecounter,\usefont,\usepackage,\vfil,\vfill,\viiipt}
\DoNotIndex{\viipt,\vipt,\vskip,\vspace}
\DoNotIndex{\wd,\xiipt,\year,\z@}
%
% \title{The bil-CV class\thanks{This document
%    corresponds to bil-CV~\fileversion 
%    dated \filedate.}}
%  \author{Ben Lavery \\ \texttt{ben.lavery@hashbang0.com}}
%
% \maketitle
% \tableofcontents
%%%%%%%%%%%%%%%%%%%%%%%%%%%%%%%%%%%%%%%%%%%%%%%%%%%%%%%%%%%%%%%%%%%%%%%%
%% Authors & Credits:                                                 %%
%%                                                                    %%
%% After years of using this CV I have lost where I sourced it from,  %%
%% which I feel awful about.  It was created as a document with       %%
%% ?editable fields? and not as a class.  All I have done is made it  %%
%% into a class for convenience, though a number of my own            %%
%% customisations have been entwined with the original over the       %%
%% years.                                                             %%
%%     -Ben Lavery, hashbang0.com / https://github.com/forquare       %%
%%                                                                    %%
%% With help from:                                                    %%
%%      www.sharelatex.com/blog/2011/03/27/how-to-write-a-latex-      %%
%%         class-file-and-design-your-own-cv.html                     %%
%%                                                                    %%
%%      www.sharelatex.com/blog/2013/06/28/how-to-write-a-latex-      %%
%%         class-file-and-design-your-own-cv.html                     %%
%%                                                                    %%
%%      www.tug.org/TUGboat/tb28-1/tb88flynn.pdf                      %%
%%                                                                    %%
%%      tex.stackexchange.com/questions/81686/pass-xkeyval-option-    %%
%%         to-class                                                   %%
%%                                                                    %%
%%      tex.stackexchange.com/questions/235094/unwanted-and-          %%
%%         unwarranted-indent-after-section                           %%
%%                                                                    %%
%%      www.latex-project.org/guides/clsguide.pdf                     %%
%%                                                                    %%
%%%%%%%%%%%%%%%%%%%%%%%%%%%%%%%%%%%%%%%%%%%%%%%%%%%%%%%%%%%%%%%%%%%%%%%%

%%%%%%%%%%%%%%%%%%%%%%%%%%%%%%%%%%%%%%%%%%%%%%%%%%%%%%%%%%%%%%%%%%%%%%%%
%%%%%%%%%%%%%%%%%%%%%%%%%% Class Particulars %%%%%%%%%%%%%%%%%%%%%%%%%%%
%%%%%%%%%%%%%%%%%%%%%%%%%%%%%%%%%%%%%%%%%%%%%%%%%%%%%%%%%%%%%%%%%%%%%%%%
% Loads a ?base class? which we can modify.
\LoadClass{article}%
\ProcessOptions%

% Class requirements and description
\NeedsTeXFormat{LaTeX2e}%
\ProvidesClass{bil-CV}[2015/03/22 Ben Lavery?s CV Class]%


%%%%%%%%%%%%%%%%%%%%%%%%%%%%%%%%%%%%%%%%%%%%%%%%%%%%%%%%%%%%%%%%%%%%%%%%
%%%%%%%%%%%%%%%%%%%%%%%%%%%%%% Optionals %%%%%%%%%%%%%%%%%%%%%%%%%%%%%%%
%%%%%%%%%%%%%%%%%%%%%%%%%%%%%%%%%%%%%%%%%%%%%%%%%%%%%%%%%%%%%%%%%%%%%%%%
\DeclareOption{10pt}{\PassOptionsToClass{10pt}{article}}%
\DeclareOption{11pt}{\PassOptionsToClass{11pt}{article}}%
\DeclareOption{12pt}{\PassOptionsToClass{12pt}{article}}%
%
\ProcessOptions%


%%%%%%%%%%%%%%%%%%%%%%%%%%%%%%%%%%%%%%%%%%%%%%%%%%%%%%%%%%%%%%%%%%%%%%%%
%%%%%%%%%%%%%%%%%%%%%%%%%% Class Particulars %%%%%%%%%%%%%%%%%%%%%%%%%%%
%%%%%%%%%%%%%%%%%%%%%%%%%%%%%%%%%%%%%%%%%%%%%%%%%%%%%%%%%%%%%%%%%%%%%%%%
% Loads a ?base class? which we can modify.
\LoadClass{article}%
\ProcessOptions%

% Class requirements and description
\NeedsTeXFormat{LaTeX2e}%
\ProvidesClass{bil-CV}[2015/03/22 Ben Lavery?s CV Class]%


%%%%%%%%%%%%%%%%%%%%%%%%%%%%%%%%%%%%%%%%%%%%%%%%%%%%%%%%%%%%%%%%%%%%%%%%
%%%%%%%%%%%%%%%%%%%%%%%%%% Required Packages %%%%%%%%%%%%%%%%%%%%%%%%%%%
%%%%%%%%%%%%%%%%%%%%%%%%%%%%%%%%%%%%%%%%%%%%%%%%%%%%%%%%%%%%%%%%%%%%%%%%

% This is a helpful package that puts math inside length specifications
% http://www.ctan.org/pkg/calc
\RequirePackage{calc}%

% Good for playing with paper sizes
% http://www.ctan.org/pkg/geometry
\RequirePackage{geometry}%

% For hyperlinks in text
% http://www.ctan.org/pkg/hyperref
\RequirePackage{hyperref}%

% To remove headers and footers
% http://www.ctan.org/pkg/fancyhdr
\usepackage{fancyhdr}%


%%%%%%%%%%%%%%%%%%%%%%%%%%%%%%%%%%%%%%%%%%%%%%%%%%%%%%%%%%%%%%%%%%%%%%%%
%%%%%%%%%%%%%%%%%%%%%%%%%%%%%% Lengths %%%%%%%%%%%%%%%%%%%%%%%%%%%%%%%%%
%%%%%%%%%%%%%%%%%%%%%%%%%%%%%%%%%%%%%%%%%%%%%%%%%%%%%%%%%%%%%%%%%%%%%%%%

% Half of the page width, useful for ?false columns?
\newlength{\halfpage}%
\setlength{\halfpage}{(\textwidth+\marginparwidth+\marginparsep)/2}%

%%%%%%%%%%%%%%%%%%%%%%%%%%%%%%%%%%%%%%%%%%%%%%%%%%%%%%%%%%%%%%%%%%%%%%%%
%%%%%%%%%%%%%%%%%%%%%%%%%%%%%%% Layout %%%%%%%%%%%%%%%%%%%%%%%%%%%%%%%%%
%%%%%%%%%%%%%%%%%%%%%%%%%%%%%%%%%%%%%%%%%%%%%%%%%%%%%%%%%%%%%%%%%%%%%%%%

% Puts the section titles on left side of page
% Section titles are in the margin
\reversemarginpar%

% Set up the margins/etc
\geometry{%
	paper=a4paper,        % Use A4 paper
	marginparwidth=20mm,  % Length of section titles (marginal note width)
	marginparsep=4mm,     % Separation between the marginal note and main text
	margin=12mm,          % margins (all)
	rmargin=16mm,%
	includemp,            % Include marginal note in textwidth calculations
}

% Stop indenting throughout entire document
\setlength{\parindent}{0mm}%

% Stop hyphens
\hyphenpenalty=10000%
\exhyphenpenalty=10000%

% Remove header and footers
\pagestyle{fancy}%
\fancyhead{}%
\fancyfoot{}%
\renewcommand{\headrulewidth}{0pt}%
\renewcommand{\footrulewidth}{0pt}%


%%%%%%%%%%%%%%%%%%%%%%%%%%%%%%%%%%%%%%%%%%%%%%%%%%%%%%%%%%%%%%%%%%%%%%%%
%%%%%%%%%%%%%%%%%%%%% New Commands & Environments %%%%%%%%%%%%%%%%%%%%%%
%%%%%%%%%%%%%%%%%%%%%%%%%%%%%%%%%%%%%%%%%%%%%%%%%%%%%%%%%%%%%%%%%%%%%%%%

% The title (name) with a horizontal rule under it
% Usage: \makeheading{name}
% Place at top of document. It should be the first thing.
\newcommand{\makeheading}[1]{%
	\hspace*{-\marginparsep minus \marginparwidth}%
	\begin{minipage}[t]{\textwidth+\marginparwidth+\marginparsep}%
		{\large \bfseries #1}\\%
		[-0.15\baselineskip]%
		\rule{\columnwidth}{1pt}%
	\end{minipage}
}%

% To add some paragraph space between lines.
% This also tells LaTeX to preferably break a page on one of these gaps
% if there is a needed pagebreak nearby.
\newcommand{\littleblankline}{%
	\vspace{1.5mm}%
	\\%
	\pagebreak[2]%
}%

% Adds a short line between ?subsections?
\newcommand{\subsectionrule}{%
	\rule{300px}{0.5px}%
	\vspace{1mm}%
	 %
}%

% Adds thinspace around an en-dash for dates containing months
\newcommand{\daterange}[2]{%
	#1\thinspace--\thinspace#2%
}%

% Add contact information in a ?nice? way
\newcommand{\contactdetails}[8]{%
	\begin{minipage}[t]{\halfpage}%
		#4\\%
		#5\\%
		#6\\%
		#7\\%
		#8%
	\end{minipage}%
	\begin{minipage}[t]{\halfpage}%
		\textit{Phone}: #1\\%
		\textit{E-mail}: \href{#2}{#2}\\%
		\textit{WWW}: \href{#3}{#3}%
	\end{minipage}%
}%

% Add the ?head? section for education/work places
\newcommand{\placementheader}[4]{%
		\textbf{#1}, {\daterange{#2}{#3}}\\%
		#4%
}%

% Add the ?head? section for education/work places
\newcommand{\placementheaderonedate}[3]{%
		\textbf{#1}, {#2}\\%
		#3%
}%

% Use a normal itemised list, but compact the space between bullets to
% ?normal? line spacing.
\newenvironment{compactlist}{%
	\vspace{-2mm}%
	\begin{itemize}%
		\setlength{\itemsep}{0pt}%
		\setlength{\parskip}{0pt}%
		\setlength{\parsep}{0pt}%
}{%
	\end{itemize}%
	\pagebreak[2]%
	\vspace{-\baselineskip}%
}%

% Use a normal itemised list, but compact the space between bullets to
% ?normal? line spacing.  Use this INSIDE other compact lists.
\newenvironment{innercompactlist}{%
	\vspace{-1mm}%
	\begin{itemize}%
		\setlength{\itemsep}{0pt}%
		\setlength{\parskip}{0pt}%
		\setlength{\parsep}{0pt}%
}{ %
	\end{itemize}%
	\pagebreak[2]%
	\vspace{-1mm}%
}%


%%%%%%%%%%%%%%%%%%%%%%%%%%%%%%%%%%%%%%%%%%%%%%%%%%%%%%%%%%%%%%%%%%%%%%%%
%%%%%%%%%%%%%%%%%%%% Renew Commands & Environments %%%%%%%%%%%%%%%%%%%%%
%%%%%%%%%%%%%%%%%%%%%%%%%%%%%%%%%%%%%%%%%%%%%%%%%%%%%%%%%%%%%%%%%%%%%%%%

% Otherwise the top of the section header will not line up with the top
% of the section.
\renewcommand{\section}[2]{%
	\pagebreak[2]%
	\vspace{\baselineskip}%
	\phantomsection%
	\addcontentsline{toc}{section}{#1}%
	\hspace{0mm}%
	\marginpar{%
		\raggedright%
		\scshape #1%
	}#2%
}%

\renewcommand{\subsection}[1]{%
	\textbf{#1}%
}%


%%%%%%%%%%%%%%%%%%%%%%%%%%%%%%%%%%%%%%%%%%%%%%%%%%%%%%%%%%%%%%%%%%%%%%%%
%%%%%%%%%%%%%%%% \end{Simple LaTeX CV Template Class} %%%%%%%%%%%%%%%%%%
%%%%%%%%%%%%%%%%%%%%%%%%%%%%%%%%%%%%%%%%%%%%%%%%%%%%%%%%%%%%%%%%%%%%%%%%